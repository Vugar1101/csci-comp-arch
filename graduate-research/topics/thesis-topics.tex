\documentclass{beamer}
\usetheme{Madrid}
\usepackage{amsmath}
\usepackage{amsfonts}
\usepackage{graphicx}
\usepackage{hyperref}

% Setting the title, author, and date
\title{2025-2026 Master Thesis Topics in OpenMP and ML/AI}
\author{John Burns, ADA University}
\date{July 16, 2025}

% Beginning the document
\begin{document}

% Creating the title slide
\begin{frame}
    \titlepage
\end{frame}

% Introducing the overview slide
\begin{frame}{Overview}
    \tableofcontents
\end{frame}

% Section for Thesis Topic 1
\section{Introduction}
\begin{frame}{About myself and my research interests}
    \begin{itemize}
	\item My PhD was in the field of large-scale (at the time!) modelling and simulation
        \item From my industry background I also have an interest in performance, reliability and scalability
        \item From my teaching I have an interest in Computer Architecture and Distributed Systems
        \item I welcome any suggestions you may have in the intersection of these fields
        \item I am very interested in OpenMP projects - from code and performance analysis through to application developement
	\item As Energy Efficiency and Optimization is so important now, why not use ARM64 cloud platforms for these research topics.
    \end{itemize}
\end{frame}

\section{Topic 1: Optimizing Deep Neural Network Training with OpenMP}
\begin{frame}{Topic 1: Optimizing Deep Neural Network Training with OpenMP}
\begin{itemize}
    \item \textbf{Objective}: Enhance the performance of deep neural network (DNN) training by parallelizing compute-intensive operations (e.g., matrix multiplications, gradient computations) using OpenMP on multi-core CPUs.
    \item \textbf{Description}: Implement and evaluate OpenMP-based parallelization strategies for DNN training in frameworks like PyTorch or TensorFlow. Focus on optimizing data parallelism and model parallelism for large-scale datasets.
    \item \textbf{Relevance}: Reduces training time for DNNs, making them more feasible for resource-constrained environments.
    \item \textbf{Methodology}: Develop parallelized versions of backpropagation and optimization algorithms, benchmark performance on multi-core systems, and compare with GPU-based implementations.
\end{itemize}
\end{frame}

\begin{frame}{Topic 1: References}
\begin{itemize}
    \item \textbf{Rakhimov, M., et al.}, "Parallel Approaches in Deep Learning: Use Parallel Computing," \textit{International Conference on Future Networks and Distributed Systems (ICFNDS)}, 2023. \url{https://doi.org/10.1145/3644713.3644738}
    \item \textbf{Zhang, Y., et al.}, "Exploiting Parallelism Opportunities with Deep Learning Frameworks," \textit{ACM Transactions on Architecture and Code Optimization}, vol. 18, no. 1, 2020. \url{https://doi.org/10.1145/3431388}

\end{itemize}
\end{frame}

\section{Topic 2: Help! Advising OpenMP Parallelization}
\begin{frame}{Topic 2: Help! Advising OpenMP Parallelization}
\begin{itemize}
    \item \textbf{Objective}: Reduce the programming burden of OpenMP annotation tasks.
    \item \textbf{Description}: Research DeepTyper-based learning architecture for advising OpenMP Annotation / Parallelization.
    \item \textbf{Relevance}: Valuable research to reduce the human workload in OpenMP development.
    \item \textbf{Methodology}: Using CodeT5+ and DeepTyper approaches.
\end{itemize}
\end{frame}

\begin{frame}{Topic 2: References}
\begin{itemize}
    \item \textbf{Pornmaneerattanatri, S., et al.,} "Automatic Parallelization with CodeT5+: A Model for Generating OpenMP Directives," \textit{International Workshop on Large Language Models and HPC}, 2024.
    \item \textbf{Shen, Y., et al.}, "A machine learning method to variable classification in OpenMP," \textit{Concurrency and Computation: Practice and Experience}, 2023. \url{https://doi.org/10.1002/cpe.7746}
    \item \textbf{Kadosh, T., et al.}, "Advising OpenMP Parallelization via A Graph-Based Approach with Transformers"
    OpenMP: Advanced Task-Based, Device and Compiler Programming (pp.3-17) 
\end{itemize}
\end{frame}

\section{Topic 3: Parallel Graph Neural Networks with OpenMP}
\begin{frame}{Topic 3: Parallel Graph Neural Networks with OpenMP}
\begin{itemize}
    \item \textbf{Objective}: Enhance the scalability of graph neural networks (GNNs) for large-scale graph data using OpenMP parallelization.
    \item \textbf{Description}: Implement OpenMP-based parallel processing for GNN operations like message passing and aggregation, targeting applications in social networks or bioinformatics.
    \item \textbf{Relevance}: GNNs are computationally expensive for large graphs; OpenMP can improve training and inference speed on multi-core systems.
    \item \textbf{Methodology}: Develop parallel GNN algorithms, test on datasets like OGB (Open Graph Benchmark), and compare performance with serial implementations.
\end{itemize}
\end{frame}

\begin{frame}{Topic 3: References}
\begin{itemize}
    \item \textbf{Meng, Z., et al.}, "OpenMP Parallelization and Optimization of Graph-Based Machine Learning Algorithms," \textit{International Workshop on OpenMP (IWOMP)}, 2016
    \item \textbf{Zhou, J., et al.}, "Graph Neural Networks: A Review of Methods and Applications," \textit{AI Open}, vol. 1, 2020. \url{https://doi.org/10.1016/j.aiopen.2021.01.001}
\end{itemize}
\end{frame}

\section{Topic 4: Energy-Efficient ML Model Training with OpenMP}
\begin{frame}{Topic 4: Energy-Efficient ML Model Training with OpenMP}
\begin{itemize}
    \item \textbf{Objective}: Develop energy-efficient ML training pipelines using OpenMP to optimize resource utilization on multi-core CPUs.
    \item \textbf{Description}: Investigate OpenMP-based parallelization to reduce energy consumption in ML tasks, focusing on dynamic thread management and workload balancing.
    \item \textbf{Relevance}: Energy efficiency is critical for sustainable AI; OpenMP can optimize CPU-based training for green computing.
    \item \textbf{Methodology}: Implement energy-aware parallel algorithms, measure energy consumption using 
PowerPoint, and compare with traditional ML training pipelines.
\end{itemize}
\end{frame}

\begin{frame}{Topic 4: References}
\begin{itemize}
    \item \textbf{Garcia, A., et al.}, "DNN Is Not All You Need: Parallelizing Non-Neural ML Algorithms on Ultra-Low-Power IoT Processors," \textit{ACM Transactions on Embedded Computing Systems}, 2023. \url{https://doi.org/10.1145/3570152}
    \item \textbf{Strubell, E., et al.}, "Energy and Policy Considerations for Deep Learning," \textit{ACM Computing Surveys}, vol. 53, no. 3, 2020. \url{https://doi.org/10.1145/3372822}

\end{itemize}
\end{frame}

\section{Topic 5: OpenMP Enhancement of R Selected Contributed Packages}
\begin{frame}{Topic 5: OpenMP Enhancement of R Selected Contributed Packages}
\begin{itemize}
    \item \textbf{Objective}:Enhance the performance of selected package using OpenMP.
    \item \textbf{Description}: Today there are over 22000 Contributed Packages in cran-r.
    \item \textbf{Relevance}: R is one of the most popular languages for data science, with 31\% 
    of data scientists regularly using it, according to a 2025 source..
    \item \textbf{Methodology}: Select a popular R contrib package, 
    (\texttt{install.packages("packageRank")}) or some other package of your choice. Compile locally and test with and without
    OpenMP directives. Benchmark performance and contribute to the community
\end{itemize}
\end{frame}

\begin{frame}{Topic 5: References}
\begin{itemize}
    \item As of July 17, 2025, there are approximately 1.1–1.3 
million R programmers globally, based on a total R user base of ~2.1–2.2 
million and a 50–60\% programmer ratio.        
    \item Start here: \texttt{install.packages("packageRank")}
\end{itemize}
\end{frame}

\begin{frame}{Conclusion}
\begin{itemize}
    \item These topics combine OpenMP's parallel computing capabilities with cutting-edge ML/AI challenges.
    \item They address performance, scalability, privacy, and sustainability in AI systems.
    \item Obviously knowledge/interest in C and Parallel Computing
    is a requirement
    \item A good place to start: OpenMP Architecture Review Board, "OpenMP API Specification: Version 5.2," 2024. \url{https://www.openmp.org/specifications/}
\end{itemize}
\end{frame}

\end{document}
